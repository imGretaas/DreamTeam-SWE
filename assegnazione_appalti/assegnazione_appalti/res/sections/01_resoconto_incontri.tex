\section{Resoconto degli incontri esplorativi con il proponente scelto}
\subsection{Incontro 1}
\begin{itemize}
\item \textbf{Data della riunione}: 5 novembre 2021
\item \textbf{Durata}: 30 minuti
\item \textbf{Partecipanti}: membri del gruppo DreamTeam,  Stefano Dindo
\item \textbf{Luogo della riunione}: Google Meet
\item \textbf{Motivo della riunione}: Meeting conoscitivo con il proponente
\subsubsection{Argomenti trattati durante l'incontro}
Nel meeting conoscitivo con il proponente abbiamo discusso dei seguenti argomenti:
	\begin{itemize}
		\item Comprendere la difficoltà del progetto,
		\item Piattaforme di sviluppo del progetto,
		\item Attività di formazione,
		\item Interfaccia grafica dell’applicazione e della web app,
		\item Registrazione degli utenti alla piattaforma.
	\end{itemize}
\end{itemize} 

\subsection{Incontro 2}
\begin{itemize}
\item \textbf{Data della riunione}: 15 novembre 2021
\item \textbf{Durata}: 45 minuti
\item \textbf{Partecipanti}: membri del gruppo DreamTeam,  Stefano Dindo,  Michele Massaro
\item \textbf{Luogo della riunione}: Google Meet
\item \textbf{Motivo della riunione}: Comunicazione della preferenza da parte del gruppo verso il capitolato proposto, definizione degli strumenti di comunicazione e discussione più approfondita sullo svolgimento del progetto
\subsubsection{Argomenti trattati durante l'incontro}
Nei giorni antecedenti l’incontro, il gruppo si è riunito per discutere più nel dettaglio in merito allo svolgimento del capitolato ed è stata stilata una lista di domande da porre al proponente.  Gli argomenti trattati durante l’incontro con il proponente sono stati:
	\begin{itemize}
		\item Acquisizione e analisi dei dati,
		\item Frequenza di aggiornamento del database,
		\item Scala su cui costruire la piattaforma.
	\end{itemize}
	Inoltre, abbiamo accordato assieme al proponente di utilizzare Slack come mezzo di comunicazione e su quali argomenti concentrare la formazione.

\end{itemize} 


